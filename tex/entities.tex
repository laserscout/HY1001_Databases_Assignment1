\section{Μοντέλο Οντοτήτων/Συσχετίσεων}

\subsection{Γενική Περιγραφή}

Οι οντότητες είναι : οι Εκδήλωση, η Τοποθεσία, η Ημερομηνία, ο Καλλιτέχνης - Διοργανωτής, η Αγορά Εισιτηρίων και η Προσβασιμότητα. Για κάθε εκδήλωση θα πρέεπι να καταγράφεται το όνομά της, το είδος της και το όνομα του καλλιτέχνη-διοργανωτή.
\\
\\
\underline{Υποθέσεις:}
\begin{itemize}[noitemsep]

\item Ο κωδικός εκδήλωσης είναι μοναδικός για κάθε εκδήλωση. Για παράδειγμα, εφόσον ο κωδικός 101 αντιστοιχεί σε μια συγκικριμένη εκδήλωση (ασχέτως καλλιτέχνη ή τοποθεσίας), την ημερομηνία 1/12/2018, τότε ο ίδιος κωδικός δεν μπορεί να είναι κωδικός καμίας άλλης εκδήλωσης.
\item Η διαφημίσεις μπορούν να γίνουν μόνο σε έναν τηλεοπτικό ή ραδιοφωνικό σταθμό για κάθε εκδήλωση. Επίσης θα υπάρχει μόνο ένα μέρος τοποθέτησης αφισών κάθε φορά.


\end{itemize}

\subsection{Καθορισμός Οντοτήτων}

Παρακάτω φαίνονται οι οντότητες της \titlos, η περιγραφή τους καθώς και κάποια γνωρίσματά τους.

\begin{center}
\begin{tabular}[]{|c | c|}
\hline
\textbf{Όνομα Οντότητας}   &  Event  \\ \hline 
\textbf{Περιγραφή}         &  Οντότητα που αποθηκεύονται οι εκδηλώσεις \\ \hline 
\textbf{Ιδιότητες}         &  Ισχυρή οντότητα \\  \hline               
\textbf{Γνωρίσματα}        &  \underline{Κωδικός εκδήλωσης} \\
           ~               &  Είδος εκδήλωσης \\
            ~              &  Ύπαρξη Εισιτηρίου \\
             ~             &  Κοινό που απευθύνεται \\
              ~            &  Σκοπός \\ 
                           &  Ημερομηνία \\
                           &  Ώρα \\
\hline
\hline
\textbf{Όνομα Οντότητας}   &  Location \\ \hline 
\textbf{Περιγραφή}         &  Οντότητα που αποθηκεύονται οι τοποθεσίες των εκδηλώσεων \\ \hline 
\textbf{Ιδιότητες}         &  Ασθενής οντότητα \\ \hline 
\textbf{Γνωρίσματα}        &  \underline{Κωδικός τοποθεσίας} \\
                           &  Όνομα \\
           ~               &  Οδός \\
             ~             &  ΤΚ\\
                           &  Εσωτερικός ή Εξωτερικός χώρος \\
                           &  Τηλέφωνο \\
                           & { \begin{tabular}[]{c|c}
                            Κάτάλογος τιμών           & μπύρα \\
                                                      & κρασί \\
                                                      & ποτό \\  
                           \end{tabular} }  
\\ \hline

\end{tabular}

\begin{tabular}[]{|c | c | } 
\hline
\textbf{Όνομα Οντότητας}   &  Artist \\ \hline 
\textbf{Περιγραφή}         &  Οντότητα που αποθηκεύονται οι καλλιτέχνες \\ \hline 
\textbf{Ιδιότητες}         &  Ισχυρή οντότητα    \\    \hline           
\textbf{Γνωρίσματα}        &  \underline{Κωδικός καλλιτέχνη}\\
                           &  Όνομα Καλλιτέχνη \\
           ~               &  Καταγωγή \\
            ~              &  Είδος \\
\hline 
\hline
\textbf{Όνομα Οντότητας}   &  Tickets \\ \hline 
\textbf{Περιγραφή}         &  Οντότητα που αποθηκεύονται οι τρόποι αγοράς εισιτηρίων \\\hline 
\textbf{Ιδιότητες}         &  Ασθενής οντότητα \\       \hline           
\textbf{Γνωρίσματα}        &  \underline{Κωδικός εκδήλωσης} \\
                          %  &  Ύπαρξη εισιτηρίου \\
                           &  Φυσικά καταστήματα προπώλησης \\
           ~               &  Ηλεκτρονικά καταστήματα προπώλησης \\
            ~              &  Εύρος τιμών \\
\hline 
\hline
\textbf{Όνομα Οντότητας}   &  Accessibility \\ \hline 
\textbf{Περιγραφή}         &  Οντότητα που αποθηκεύονται οι τρόποι πρόσβασης στην τοποθεσια \\ \hline 
\textbf{Ιδιότητες}         &  Ασθενής οντότητα \\  \hline                 
\textbf{Γνωρίσματα}        &  \underline{Κωδικός Τοποθεσίας} \\
                           &  Ύπαρξη χώρου στάθμευσης\\
            ~              &  Ύπαρξη κοντινών στάσεων \\
             ~             &  Ύπαρξη υποδομών για ΑΜΕΑ \\
                           & { \begin{tabular}[]{c|c}
                             Ύπαρξη τοποθεσιών με μισθωμένα ΜΜΜ           & τοποθεσία \\
                                                                         & ώρα \\ 
                           \end{tabular} }  
\\ \hline
\hline
\textbf{Όνομα Οντότητας}   &  Promotion \\ \hline 
\textbf{Περιγραφή}         &  Οντότητα που αποθηκεύονται οι τρόποι προώθησης της εκδήλωσης \\ \hline 
\textbf{Ιδιότητες}         &  Ασθενής οντότητα \\  \hline                 
\textbf{Γνωρίσματα}        &  \underline{Κωδικός Εκδήλωσης} \\
                           &  Ραδιοφωνικοί σταθμοί \\
            ~              &  Τηλεοπτικοί σταθμοί \\
             ~             &  Τοποθεσίες αφισών \\
                           & { \begin{tabular}[]{c|c}
                             Διαδικτυακή διαφήμηση & Κοινωνικά δίκτυα \\
                                                   & Ψηφιακές εφημερίδες \\
                                                   & Διάφορες ιστοσελίδες\\ 
                           \end{tabular} }  
\\ \hline
\hline
\textbf{Όνομα Οντότητας}   &  Communication \\ \hline 
\textbf{Περιγραφή}         &  Οντότητα που αποθηκεύονται οι τρόποι επικοινωνίας \\ \hline 
\textbf{Ιδιότητες}         &  Ασθενής οντότητα \\  \hline                 
\textbf{Γνωρίσματα}        &  \underline{Κωδικός Εκδήλωσης} \\
                           &  \underline{Όνομα Καλλιτέχνη} \\
            ~              &  Όνομα εταιρίας παραγωγής \\
             ~             &  email \\
                           &  Τηλέφωνο \\
\\ \hline
\end{tabular}
\end{center}


\subsection{Καθορισμός Συσχετίσεων}

Παρακάτω αναφέρονται οι συσχετίσεις της βάσης δεδομένων \titlos

\begin{tabular}[]{|p{4cm}|p{10cm}|}
  \hline
  \textbf{Όνομα Συσχέτισης} & Event\_Has\_Artist\\ \hline
  \textbf{Περιγραφή} & Κάθε εκδήλωση πρέπει να έχει 1 καλλιτέχνη\\ \hline
  \textbf{Ιδιότητες} & Has-A \{αναφέρετε αν είναι Is-A και αν είναι
                       Αναδρομική, Προσδιορίζουσα, Τριαδική\} \\ \hline
  \textbf{Λόγος πληθικότητας} & n:1 \\ \hline
  \textbf{Συμμετοχή} & Ολική Συμμετοχή του Event \\ \cline{2-2}
                     & Μερική Συμμετοχή του Artist \\ \hline
  \textbf{Γνωρίσματα} & - \\ \hline
\end{tabular}


\begin{tabular}[]{|p{4cm}|p{10cm}|}
  \hline
  \textbf{Όνομα Συσχέτισης} & Event\_Has\_Location\\ \hline
  \textbf{Περιγραφή} & Κάθε εκδήλωση πρέπει να έχει 1 τοποθεσία\\ \hline
  \textbf{Ιδιότητες} & Has-A \{αναφέρετε αν είναι Is-A και αν είναι
                       Αναδρομική, Προσδιορίζουσα, Τριαδική\} \\ \hline
  \textbf{Λόγος πληθικότητας} & n:1 \\ \hline
  \textbf{Συμμετοχή} & Ολική Συμμετοχή του Event \\ \cline{2-2}
                     & Μερική Συμμετοχή του Location \\ \hline
  \textbf{Γνωρίσματα} & - \\ \hline
\end{tabular}


\begin{tabular}[]{|p{4cm}|p{10cm}|}
  \hline
  \textbf{Όνομα Συσχέτισης} & Event\_Has\_Date\\ \hline
  \textbf{Περιγραφή} & Κάθε εκδήλωση πρέπει να έχει 1 ημερομηνία\\ \hline
  \textbf{Ιδιότητες} & Has-A \{αναφέρετε αν είναι Is-A και αν είναι
                       Αναδρομική, Προσδιορίζουσα, Τριαδική\} \\ \hline
  \textbf{Λόγος πληθικότητας} & n:1 \\ \hline
  \textbf{Συμμετοχή} & Ολική Συμμετοχή του Event \\ \cline{2-2}
                     & Μερική Συμμετοχή του Date \\ \hline
  \textbf{Γνωρίσματα} & - \\ \hline
\end{tabular}


\begin{tabular}[]{|p{4cm}|p{10cm}|}
  \hline
  \textbf{Όνομα Συσχέτισης} & Event\_Has\_Date\\ \hline
  \textbf{Περιγραφή} & Κάθε εκδήλωση πρέπει να έχει 1 ημερομηνία\\ \hline
  \textbf{Ιδιότητες} & Has-A \{αναφέρετε αν είναι Is-A και αν είναι
                       Αναδρομική, Προσδιορίζουσα, Τριαδική\} \\ \hline
  \textbf{Λόγος πληθικότητας} & n:1 \\ \hline
  \textbf{Συμμετοχή} & Ολική Συμμετοχή του Event \\ \cline{2-2}
                     & Μερική Συμμετοχή του Date \\ \hline
  \textbf{Γνωρίσματα} & - \\ \hline
\end{tabular}

\begin{tabular}[]{|p{4cm}|p{10cm}|}
  \hline
  \textbf{Όνομα Συσχέτισης} & Event\_Has\_Tickets\\ \hline
  \textbf{Περιγραφή} & Κάθε εκδήλωση πρέπει να έχει μέρη που πωλούνται εισιτήρια\\ \hline
  \textbf{Ιδιότητες} & Has-A \{αναφέρετε αν είναι Is-A και αν είναι
                       Αναδρομική, Προσδιορίζουσα, Τριαδική\} \\ \hline
  \textbf{Λόγος πληθικότητας} & n:1 \\ \hline
  \textbf{Συμμετοχή} & Ολική Συμμετοχή του Event \\ \cline{2-2}
                     & Μερική Συμμετοχή του Tickets\\ \hline
  \textbf{Γνωρίσματα} & - \\ \hline
\end{tabular}

\begin{tabular}[]{|p{4cm}|p{10cm}|}
  \hline
  \textbf{Όνομα Συσχέτισης} & Location\_Has\_Accessibility\\ \hline
  \textbf{Περιγραφή} & Κάθε τοποθεσία πρέπει να έχει τρόπους πρόσβασης\\ \hline
  \textbf{Ιδιότητες} & Has-A \{αναφέρετε αν είναι Is-A και αν είναι
                       Αναδρομική, Προσδιορίζουσα, Τριαδική\} \\ \hline
  \textbf{Λόγος πληθικότητας} & n:1 \\ \hline
  \textbf{Συμμετοχή} & Ολική Συμμετοχή του Location \\ \cline{2-2}
                     & Μερική Συμμετοχή του Accessibility\\ \hline
  \textbf{Γνωρίσματα} & - \\ \hline
\end{tabular}

\begin{tabular}[]{|p{4cm}|p{10cm}|}
  \hline
  \textbf{Όνομα Συσχέτισης} & Event\_Has\_Communication\\ \hline
  \textbf{Περιγραφή} & Κάθε εκδήλωση πρέπει να έχει τρόπους επικοινωνίας\\ \hline
  \textbf{Ιδιότητες} & Has-A \{αναφέρετε αν είναι Is-A και αν είναι
                       Αναδρομική, Προσδιορίζουσα, Τριαδική\} \\ \hline
  \textbf{Λόγος πληθικότητας} & n:n \\ \hline
  \textbf{Συμμετοχή} & Ολική Συμμετοχή του Event \\ \cline{2-2}
                     & Μερική Συμμετοχή του Communication \\ \hline
  \textbf{Γνωρίσματα} & - \\ \hline
\end{tabular}

\begin{tabular}[]{|p{4cm}|p{10cm}|}
  \hline
  \textbf{Όνομα Συσχέτισης} & Event\_Has\_Promotion\\ \hline
  \textbf{Περιγραφή} & Κάθε εκδήλωση πρέπει να έχει τρόπους προώθησης\\ \hline
  \textbf{Ιδιότητες} & Has-A \{αναφέρετε αν είναι Is-A και αν είναι
                       Αναδρομική, Προσδιορίζουσα, Τριαδική\} \\ \hline
  \textbf{Λόγος πληθικότητας} & n:n \\ \hline
  \textbf{Συμμετοχή} & Ολική Συμμετοχή του Event \\ \cline{2-2}
                     & Μερική Συμμετοχή του Promotion \\ \hline
  \textbf{Γνωρίσματα} & - \\ \hline
\end{tabular}

\subsection{Διάγραμμα Οντοτήτων/Συσχετίσεων}

\{Δείξτε το διάγραμμα Ο/Σ για τη βάση. Το διάγραμμα μπορείτε να το
κατασκευάσετε σε πρόγραμμα της επιλογής σας, ωστόσο θα πρέπει να
ακολουθεί το συμβολισμό Chen (δηλαδή οντότητες ως παραλληλόγραμμα,
συσχετίσεις ως ρόμβοι, διπλή γραμμή για υποχρεωτική συμμετοχή, κτλ.)\}

Παράδειγμα για τη FlightsDB:
\begin{figure}[H]
  \centering
  \includegraphics[width=\linewidth]{entities.png}
  \caption{Διάγραμμα Οντοτήτων/Συσχετίσεων}
\end{figure}


%%% Local Variables:
%%% mode: latex
%%% TeX-master: "main"
%%% End:
