
\section{Κατηγορίες Χρηστών και απαιτήσεις τους}

Στην συγκεκριμένη εφαρμογή και κατ' επέκταση η βάση δεδομένων θα έχει
έναν διαχειριστή και τρεις χρήστες, τον "Διοργανωτή" τον "Μη
Εγγεγραμμένο Χρήστη" και τον "Χρήστη". Μόνο οι τρεις χρήστες ορίζονται
παρακάτω μιας και ο διαχειριστής της εφαρμογής και της βάσης δεδομένων
θα εκτελεί ενέργειες με αυτόνομο τρόπο πέρα των πλαισίων της
εφαρμογής.

\underline{Διοργανωτής:}

Ο Διοργανωτής, μετά από εγγραφή του στο σύστημα, η οποία εγκρίνεται
από τον διαχειριστή, πρέπει να έχει την δυνατότητα να εκτελεί όλες τις
απαραίτητες ενέργειες έτσι ώστε να καταχωρεί όλες τις απαραίτητες
πληροφορίες μιας εκδήλωσης όπως και να έχει πρόσβαση στην λίστα αγορών
για τις εκδηλώσεις όπου διαχειρίζεται. Αναλυτικά:
\begin{itemize}[noitemsep]
\item Προσθήκη νέας τοποθεσίας διεξαγωγής
\item Προσθήκη νέων σημείων προπόλησης
\item Προσθηκη νέας εκδήλωσης
\item Προβολή λίστας αγορών εκδήλωσης όπου οργανώνει
\end{itemize}

\underline{Μη εγγεγραμμένος χρήστης}

Ο μη εγγεγραμμένος χρήστης έχει την δυνατότητα να προβάλει με διάφορα
κριτήρια εύρεσης τις μελλοντικές εκδηλώσεις και να πραγματοποιήσει
εγγραφή
\begin{itemize}[noitemsep]
\item Πρόσβαση σε δεδομένα που αφορούν τις εκδηλώσεις, μετά απο
  σχετική αναζήτηση.
\item Εγγραφή χρήστη
\end{itemize}

\underline{Χρήστης:}

Ο Χρήστης μετά από εγγραφή του, η οποία ολοκληρώνεται αυτόματα, έχει
την επιπλέον δυνατότητα, πέρα του μη εγγεγραμμένου χρήστη, να εκτελεί
αγορά εισιτήριων για τις εκδηλώσεις που το υποστηρίζουν, όπως και να
αποθηκεύει εκδηλώσεις που των ενδιαφέρουν για να τις δει
αργότερα. Αναλυτικά:
\begin{itemize}[noitemsep]
\item Προσθήκη νέας κάρτας πληρωμής
\item Αγορά εισιτήριου εκδήλωσης
\item Προσθήκη και αφαίρεση εκδήλωσης στην λίστα ενδιαφερομένων
\item Προβολή εκδηλώσεων στην λίστα ενδιαφερομένων
\end{itemize}


%%% Local Variables:
%%% mode: latex
%%% TeX-master: "main"
%%% End:
