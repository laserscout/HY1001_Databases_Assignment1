\section{Εισαγωγή}

\subsection{Σκοπός Εφαρμογής}

\par Οι σύγχρονες πόλεις, καθημερινά, δίνουν την δυνατότητα σε πολλούς καλλιτέχνες και μη,
να προβάλουν την δουλειά τους μέσω εκθέσεων, συναυλιών ή άλλων εκδηλώσεων. Επίσης, καθημερινά διάφοροι οργανισμοί και ομάδες διοργανώνουν διάφορες δραστηριότητες προς υποστήριξη και ενημέρωση του κόσμου για τον σκοπό τους. 
\par Αποτέλεσμα όλων αυτών είναι, στην σημερινή κοινωνία, τα δρώμενα που λαμβάνουν χώρα καθημερινά να είναι πολυπληθή. Έτσι είναι απαραίτητη μια βάση δεδομένων που θα περιέχει δεδομένα για όλες αυτές τις εκδηλώσεις έτσι ώστε να μπορούν να καταγράφονται και ο καθένας να μπορεί, προσπελάζοντας τη βάση, να βρίσκει τις δραστηριότητες που τον ενδιαφέρουν με βάση χαρακτηριστικά τους.
\par Συγκικριμένα, στη δική μας βάση EventDB, εκος από τοποθεσία, είδος και ημερομηνία της εκδήλωσης, ο χρήστης θα μπορεί να αναζητήσει και την προσβασιμότητα της τοποθεσίας, τους τρόπους αγοράς εισιτηρίων , σε ποιο κοινό απευθύνεται κτλ. 

\subsection{Περιγραφή Εφαρμογής}

\par Για την βάση \titlos, τα δεδομένα, που θα αποθηκέυονται είναι το όνομα των εκδηλώσεων, το είδος τους,  οι ημερομηνίες διεξαγωγής τους, η τοποθεσία που πραγματοποιούνται κτλ. Τη βάση θα μπορεί αν την χρησιμοποιήσει ο οποιοσδήποτε, αρκεί να έχει πρόσβαση σε αυτήν μέσω της εφαρμογής .Επίσης, όποιος θα ήθελε η εκδήλωσή του να δημοσιοποιηθεί, θα μπορεί με μήνυμα στους διαχειριστές της σελίδας, να στείλει τα στοιχεία της, και εφόσον το μήνυμα εγκριθεί, να ανέβει η εκδήλωση στην εφαρμογή. Σε αυτήν την περίπτωση , ο διοργανωτής μπορεί να στείλει όσο περισσότερες λεπτομέριες θέλει ο ίδιος, απαραίτητα όμως είναι τα στοιχεία ονόματος της εκδήλωσης, ημερομηνίας, τοποθεσίας και είδους.

\subsection{Απαιτήσεις Εφαρμογής σε Δεδομένα}

\par Για την βάση \titlos, αναμένεται να έχουμε ~1050 κωδικούς εκδηλώσεων (πχ για έναν μήνα) , που σημαίνει ~35 κωδικοί εκδηλώσεων κάθε μέρα. Επίσης, αναμένεται οι ~20 να είναι μουσικής, οι ~25 να είναι κάτα μέσο όρο απογευματινές ώρες κτλ



%%% Local Variables:
%%% mode: latex
%%% TeX-master: "main"
%%% End:
