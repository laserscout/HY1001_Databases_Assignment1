\section{Εισαγωγή}

\subsection{Σκοπός Εφαρμογής}

Οι σύγχρονες πόλεις, καθημερινά, δίνουν την δυνατότητα σε πολλούς
καλλιτέχνες και μη, να προβάλουν την δουλειά τους μέσω εκθέσεων,
συναυλιών ή άλλων εκδηλώσεων. Επίσης, καθημερινά διάφοροι οργανισμοί
και ομάδες διοργανώνουν διάφορες δραστηριότητες προς υποστήριξη και
ενημέρωση του κόσμου για τον σκοπό τους.

Αποτέλεσμα όλων αυτών είναι, στην σημερινή κοινωνία, τα δρώμενα που
λαμβάνουν χώρα καθημερινά να είναι πολυπληθή. Έτσι είναι απαραίτητη
μια εφαρμογή όπου θα περιέχει πληροφορίες για όλες αυτές τις
εκδηλώσεις έτσι ώστε να μπορούν οι ενδιαφερόμενοι να βρίσκουν τις
δραστηριότητες που τους ενδιαφέρουν. Μία τέτοια εφαρμογή απαιτεί μία
βάση δεδομένων για την αποθήκευση, προσπέλαση και επεξεργασία των
πληροφοριών κάθε εκδήλωσης λόγο του μεγάλου όγκου της πληροφορίας
αυτής και την ανάγκη για παράλληλη επεξεργασία δεδομένων από πολλούς
χρήστες.

\subsection{Περιγραφή Εφαρμογής}

Συγκεκριμένα, στη δική μας εφαρμογή, εκος από τοποθεσία, είδος και
ημερομηνία της εκδήλωσης, ο χρήστης θα μπορεί να αγοράσει εισιτήρια
εκδηλώσεων ή να βρει φυσικά καταστήματα προπώλησης, να αποθηκεύσει
εκδηλώσεις που τον ενδιαφέρουν ώστε να τις δει αργότερα και άλλα. Όλα
αυτά είναι εφικτά λόγο της προσεκτικής σχεδίασης της βάσης δεδομένων
πίσω από την εφαρμογή

Για την βάση \titlos, τα δεδομένα, που θα αποθηκεύονται είναι το όνομα
των εκδηλώσεων, το είδος τους, οι ημερομηνίες διεξαγωγής τους, η
τοποθεσία που πραγματοποιούνται κτλ. Τη βάση θα μπορεί αν την
χρησιμοποιήσει ο οποιοσδήποτε, αρκεί να έχει πρόσβαση σε αυτήν μέσω
της εφαρμογής μας. Επίσης, όποιος
θα ήθελε η εκδήλωσή του να δημοσιοποιηθεί, θα μπορεί συμπληρώνοντας
μια φόρμα εγγραφής να αποκτήσει πρόσβασή στην πλατφόρμα δημιουργίας
εκδήλωσης και να προστεθεί η εκδήλωση του στον ιστότοπο.

\subsection{Απαιτήσεις Εφαρμογής σε Δεδομένα}

Για την βάση \titlos, αναμένεται να έχουμε ~1050 κωδικούς εκδηλώσεων
(πχ για έναν μήνα) , που σημαίνει ~35 κωδικοί εκδηλώσεων κάθε
μέρα. Επίσης, αναμένεται οι ~20 να είναι μουσικής, οι ~25 να είναι
κάτα μέσο όρο απογευματινές ώρες κτλ



%%% Local Variables:
%%% mode: latex
%%% TeX-master: "main"
%%% End:
