\section{Παραδείγματα}

\subsection{Παραδείγματα Πινάκων}

Παράδειγμα για τον πίνακα Μουσική\_Εκδήλωση της EventsDB:

\begin{table}[H]
  \centering
  \footnotesize
\begin{tabular}[c]{|p{2cm}|p{2.5cm}|p{2.5cm}|p{2cm}|p{3.5cm}|}
  \hline
  Κωδικός\_ Εκδήλωσης & Όνομα                           & Ύπαρξη\_Θέσεων\_ Καθήμενων & Είδος                       & Opening\_Act            \\ \hline
  8053                & Ημισκούμπρια - 23rd Anniversary Live & Όχι                         & Hip hop Live                & DJ Πρύτανης warm up set \\ \hline
  8055                & Συναυλία Θάνου Μικρούτσικου     & Όχι                         & Έντεχνο τραγούδι - Συναυλία & Μίλτος Πασχαλίδης       \\ \hline
\end{tabular}
\end{table}

Εκτίμηση για τον αριθμό των εγγραφών: \textasciitilde 40000

Παράδειγμα για τον πίνακα Αθλητική\_Εκδήλωση της EventsDB:

\begin{table}[H]
  \centering
  \footnotesize
  \begin{tabular}{|p{3cm}|p{4cm}|p{3cm}|p{3cm}|}
  \hline
  Κωδικός\_Εκδήλωσης & Όνομα & Ύπαρξη\_Θέσεων\_VIP & Άθλημα \\ \hline
  4444 & Τελικός Πρωταθλήματος ΠΑΟΚ - Παναθηναϊκός & Ναι & Μπάσκετ \\ \hline
  1235 & Προβολή Live Τελικού Roldand Garros & Ναι & Τένις \\ \hline
  6969 & Manos' Beerpong Challenge X Erasmus & Όχι & Beerpong \\ \hline
\end{tabular}
\end{table}
  
Εκτίμηση για τον αριθμό των εγγραφών: \textasciitilde 40000

Παράδειγμα για τον πίνακα Θέατρο της EventsDB:

\begin{table}[H]
  \centering
  \footnotesize
  \begin{tabular}{|p{3cm}|p{4cm}|p{3cm}|p{3cm}|}
  \hline
  Κωδικός\_Εκδήλωσης & Όνομα & Ύπαρξη\_Θέσεων\_VIP & Διάρκεια \\ \hline
  7777 & Οι Βάκχες του Ευρυπίδη & Όχι & 114 \\ \hline
  0666 & Stand up Comedy by George Carlin & Ναι & 74 \\ \hline
\end{tabular}
\end{table}
  
Εκτίμηση για τον αριθμό των εγγραφών: \textasciitilde 40000

Παράδειγμα για τον πίνακα Εισιτήριο της EventsDB:

\begin{table}[H]
  \centering
  \footnotesize
  \begin{tabular}{|c|c|c|}
  \hline
  Κωδικός\_Εκδήλωσης & Τύπος\_Εισιτηρίου & Τιμή \\ \hline
  7777 & Φοιτητικό & 14 \\ \hline
  7777 & Ενηλίκων & 28 \\ \hline
  8053 & Ενηλίκων & 12 \\ \hline
  8055 & Φοιτητικό & 14 \\ \hline
  8055 & Ενηλίκων & 18 \\ \hline
  4444 & Ενηλίκων & 10 \\ \hline
  1235 & Ενηλίκων & 4 \\ \hline
  0666 & Ενηλίκων & 30 \\ \hline
  0666 & Υπερήλικων & 20 \\ \hline
  6969 & Ενηλίκων & 5 \\ \hline
\end{tabular}
\end{table}
  
Εκτίμηση για τον αριθμό των εγγραφών: \textasciitilde 40000

Παράδειγμα για τον πίνακα Φυσικό\_Σημείο\_Προπώλησης της EventsDB:

\begin{table}[H]
  \centering
  \footnotesize
  \begin{tabular}{|p{3cm}|p{4cm}|p{3cm}|p{3cm}|}
  \hline
  Κωδικός\_Σημείου & Όνομα & Τηλέφωνο & Διεύθυνση \\ \hline
  12345 & Καταστήματα Public Τσιμισκή & 2310227288 & Τσιμισκή 24 \\ \hline
  14310 & Γραμματεία Τμήματος ΗΜΜΥ - Σούλα Γκλάμουρους & 2310 666 666 & Θεσσαλονίκη 54124 \\ \hline
\end{tabular}
\end{table}
  
Εκτίμηση για τον αριθμό των εγγραφών: \textasciitilde 40000

Παράδειγμα για τον πίνακα Προπώληση της EventsDB:

\begin{table}[H]
  \centering
  \footnotesize
  \begin{tabular}{|c|c|}
  \hline
  Κωδικός\_Σημείου & Κωδικός\_Εκδήλωσης \\ \hline
  12345 & 7777 \\ \hline
  12345 & 0666 \\ \hline
  12345 & 8053 \\ \hline
  12345 & 8055 \\ \hline
  14310 & 4444 \\ \hline
  14310 & 1235 \\ \hline
  14310 & 6969 \\ \hline
\end{tabular}
\end{table}
  
Εκτίμηση για τον αριθμό των εγγραφών: \textasciitilde 40000

Παράδειγμα για τον πίνακα Τοποθεσία της EventsDB:

\begin{table}[H]
  \centering
  \footnotesize
  \begin{tabular}{|p{1.8cm}|p{2cm}|p{1.6cm}|p{1.6cm}|p{2.5cm}|l}
  \hline
  Κωδικός\_Τοποθεσίας & Όνομα & Εσωτερικός\_ Χώρος & Τηλέφωνο & Διεύθυνση
    & \ldots \\ \hline
  132435 & WE - Sports \& culture facility & Ναι & 2310284700 & 3ης Σεπτεμβρίου 3 & \ldots \\ \hline
  532413 & Paok Sports Arena & Ναι & 2310192600 & Πυλαία-Χορτιάτης 555 35 &\ldots \\ \hline
  \end{tabular}
  \begin{tabular}{r|p{2cm}|p{2cm}|p{2cm}|p{2cm}|}
  \hline
    \ldots & Ύπαρξη\_ Υποδομών\_ΑΜΕΑ & Τιμή\_ Μπύρας & Τιμή\_ Κρασιού
    & Τιμή\_ Ποτού \\ \hline
    \ldots & Όχι & 4 & 6 & 7 \\ \hline
    \ldots & Ναι & NULL & NULL & NULL \\ \hline
\end{tabular}
\end{table}
  
Εκτίμηση για τον αριθμό των εγγραφών: \textasciitilde 40000

Παράδειγμα για τον πίνακα Εκδήλωση της EventsDB:

\begin{table}[H]
  \centering
  \footnotesize
  \begin{tabular}{|p{1.6cm}|p{2.8cm}|p{1.5cm}|l|l|l}
  \hline
  Κωδικός\_ Εκδήλωσης & Όνομα                               
    & Ύπαρξη\_ Εισιτηρίου & Κοινό              & Περιγραφή & \ldots \\ \hline
  6969               & Manos' Beerpong Challenge X Erasmus 
    & Ναι                & Θαρραλέοι Φοιτητές & Try to not get
                                                shitfaced & \ldots  \\ \hline
  8053               & Ημισκούμπρια - 23rd Anniversary Live
    & Ναι            & Χιουμορίστες και Χιπχοπάδες & Πάμε όλοι μαζί σε μια παραλία & \ldots  \\ \hline
  8055               & Συναυλία Θάνου Μικρούτσικου & Ναι
    & Ελεύθεροι και χωρισμένοι & Συναυλία & \ldots  \\ \hline
  4444               & Τελικός Πρωταθλήματος ΠΑΟΚ - Παναθηναϊκός
    & Ναι            & Άρρωστα παόκια κυρίως & Μην τα σπάσετε όλα όλα & \ldots \\hline
  1235               & Προβολή Live Τελικού Roldand Garros 
    & Ναι            & Φίλοι Αντισφαίρισης & Μπύρες στη μισή τιμή   \\ \hline
  7777               & Οι Βάκχες του Ευρυπίδη & Ναι
    & Ψαγμένα τυπάκια & Λε κουλτουρ Υψηλό & \ldots \\ \hline
  0666               & Stand up Comedy by George Carlin
    & Ναι            & Μη παρεξηγησιάρηδες & One Last HBO Special & \ldots \\ \hline
  \end{tabular}
   \begin{tabular}{r|l|l|l|l|l|}
  \hline
  \ldots & Ημερομηνία         & Ώρα\_Έναρξης & Κωδικός\_Τοποθεσίας & Κωδικός\_Ερμηνευτή & Κωδικός\_Διοργανωτή \\ \hline
  \ldots & 23 Δεκεμβρίου 2018 & 21:00        & 132435              & NULL               & 72150               \\ \hline
  \ldots & 20 Μαρτίου 2019    & 21:00        & 532413              & 122333             & 11888               \\ \hline
  \ldots & 18 Ιουνίου 2019    & 20:30        & 532413              & 555551             & 11888               \\ \hline
  \ldots & 10 Μαρτίου 2019    & 19:30        & 532413              & 444444             & 11888               \\ \hline
  \ldots & 29 Μαϊου 2019      & 16:30        & 132435              & NULL               & 72150               \\ \hline
  \ldots & 19 Δεκέμβρη 2018   & 20:30        & 532413              & 192837             & 11888               \\ \hline
  \ldots & 25 Μαρτίου 2019    & 00:00        & 532413              & 382957             & 72510               \\ \hline
\end{tabular}
\end{table}
  
Εκτίμηση για τον αριθμό των εγγραφών: \textasciitilde 40000

Παράδειγμα για τον πίνακα Καλλιτέχνης\_Ομάδα της EventsDB:

\begin{table}[H]
  \centering
  \footnotesize
  \begin{tabular}{|l|l|l|}
  \hline
  Κωδικός\_Ερμηνευτή & Καταγωγή & Όνομα \\ \hline
  122333 & Αθήνα - Ελλάδα & Μετζέλος, Μιρθιδάτης, Dj Πρύτανης \\ \hline
  555551 & Ελλάδα & Θάνος Μικρούτσικος \\ \hline
  444444 & Ελλάδα & Ομάδα Μπάσκετ Ενηλίκων ΠΑΟΚ \\ \hline
  382957 & USA cemetery & George Carlin's Ghost  \\ \hline
  192837 & Ελλάδα & Ανώτερη Δραματική Σχολή Θεσσαλονίκης \\ \hline
\end{tabular}
\end{table}
  
Εκτίμηση για τον αριθμό των εγγραφών: \textasciitilde 40000

Παράδειγμα για τον πίνακα καλλιτέχνης της EventsDB:

\begin{table}[H]
  \centering
  \footnotesize
  \begin{tabular}{|l|l|l|}
  \hline
  Κωδικός\_Ερμηνευτή & Είδος & Ημερομηνία\_Γέννησης \\ \hline
  555551 & Έντεχνο & 1947 \\ \hline
  122333 & Ελληνικό Hip Hop & 1975 \\ \hline
\end{tabular}
\end{table}
  
Εκτίμηση για τον αριθμό των εγγραφών: \textasciitilde 40000

Παράδειγμα για τον πίνακα Ομάδα της EventsDB:

\begin{table}[H]
  \centering
  \footnotesize
  \begin{tabular}{|l|l|}
  \hline
  Κωδικός\_Ερμηνευτή & Όνομα\_Υπευθύνου \\ \hline
  444444 & Ιβάν Σαββίδης \\ \hline
  382957 & Ηλίας Ψινάκης \\ \hline
  192837 & Ανδρέας Βουτσινάς \\ \hline
\end{tabular}
\end{table}
  
Εκτίμηση για τον αριθμό των εγγραφών: \textasciitilde 40000

Παράδειγμα για τον πίνακα Διοργανωτής της EventsDB:

\begin{table}[H]
  \centering
  \footnotesize
  \begin{tabular}{|l|l|l|l|l|}
  \hline
  Κωδικός\_Διοργανωτή & Όνομα\_Εταιρίας & Τηλέφωνο & Email & Password \\ \hline
  72150 & Party Animal Events & 6981811474 & manoszisis@yahoo.gr & 3ebb9abf12d5b17\ldots \\ \hline
  11888 & Culture AE @ 6979695949 & og34582@mhrit.net & 2p3oroh2c3ri23i\ldots \\ \hline
\end{tabular}
\end{table}
  
Εκτίμηση για τον αριθμό των εγγραφών: \textasciitilde 40000

Παράδειγμα για τον πίνακα Χρήστης της EventsDB:

\begin{table}[H]
  \centering
  \footnotesize
  \begin{tabular}{|l|l|l|l|}
  \hline
  Κωδικός\_Χρήστη & Ονοματεπώνυμο & Email & Password \\ \hline
  006689 & Φρανγκίσκος Μπλανίνγκιος & pinkypromises@gmail.com &
                                                                82a545b150da45c\ldots \\ \hline
  008055 & Χρυσηίδα Τοντόροβιτς & cultureoverload@gmail.com & 2cur239vj293d\ldots \\ \hline
\end{tabular}
\end{table}
  
Εκτίμηση για τον αριθμό των εγγραφών: \textasciitilde 40000

Παράδειγμα για τον πίνακα Κάρτα της EventsDB:

\begin{table}[H]
  \centering
  \footnotesize
  \begin{tabular}{|l|l|l|l|}
  \hline
  Αριθμός\_Κάρτας & Αριθμός\_Ασφαλείας & Κωδικός\_Χρήστη & Διεύθυνση \\ \hline
  1234 5678 9999 & 489 & 006689 & Προβληματικού 6 \\ \hline
  9999 8765 4321 & 987 & 008055 & κωμωδίας 21 \\ \hline
\end{tabular}
\end{table}
  
Εκτίμηση για τον αριθμό των εγγραφών: \textasciitilde 40000

Παράδειγμα για τον πίνακα Αγορά της EventsDB:

\begin{table}[H]
  \centering
  \footnotesize
  \begin{tabular}{|l|l|l|}
  \hline
  Κωδικός\_Εκδήλωσης & Κωδικός\_Χρήστη & Τύπος\_Εισιτηρίου \\ \hline
  0666 & 006689 & Υπερήλικων \\ \hline
  8055 & 008055 & Εφηβικό \\ \hline
  6969 & 006689 & Ενηλίκων \\ \hline
  4444 & 008055 & Ενηλίκων \\ \hline
  
\end{tabular}
\end{table}
  
Εκτίμηση για τον αριθμό των εγγραφών: \textasciitilde 40000

Παράδειγμα για τον πίνακα Ενδιαφέρον της EventsDB:

\begin{table}[H]
  \centering
  \footnotesize
  \begin{tabular}{|l|l|}
  \hline
  Κωδικός\_Εκδήλωσης & Κωδικός\_Χρήστη \\ \hline
  4444 & 008055 \\ \hline
  1235 & 006689 \\ \hline
  6969 & 006689 \\ \hline
  0666 & 006689 \\ \hline
  8055 & 008055 \\ \hline
\end{tabular}
\end{table}
  
Εκτίμηση για τον αριθμό των εγγραφών: \textasciitilde 40000

\subsection{Παραδείγματα Ερωτημάτων}

Τα κυριότερα ερωτήματα αφορούν την προβολή στοιχείων εκδηλώσεων με
διάφορα κριτήρια. Οποιοσδήποτε συνδυασμός γνωρισμάτων μπορεί να
προβληθεί. Στην παρακάτω περίπτωση θα προβληθεί μόνο το όνομα της
εκδήλωσης, η ημερομηνία και είτε το όνομα της τοποθεσίας διεξαγωγής
είτε το όνομα του καλλιτέχνη.  Κάποια κριτήρια αναζήτησής εκδηλώσεων
είναι βάση του ονόματος του τοποθεσίας (\ref{eq1}), όνομα του
καλλιτέχνη (\ref{eq2}), ή συγκεκριμένου τύπου εκδήλωσης (\ref{eq3}).

\begin{equation} \label{eq1}
\begin{split}
&A \leftarrow \text{Εκδήλωση} \bowtie
\text{Καλλιτέχνης-Ομάδα} \bowtie
\sigma_{<\text{Όνομα\_τοποθεσίας} = \text{"Τα Ξύδια"}>}
\text{Τοποθεσία}
\\
&\Pi_{<\text{Όνομα\_εκδήλωσης, Ημερομηνία,Όνομα}>}A
\end{split}
\end{equation}

\begin{equation} \label{eq2}
\begin{split}
&A \leftarrow \text{Εκδήλωση} \bowtie
\text{Τοποθεσία} \bowtie
\sigma_{<\text{Όνομα} = \text{"Γιάννης Μπουζούκης"}>}
\text{Καλλιτέχνης-Ομάδα}
\\
&\Pi_{<\text{Όνομα\_εκδήλωσης, Ημερομηνία,Όνομα\_Τοποθεσίας}>}A
\end{split}
\end{equation}

\begin{equation} \label{eq3}
\begin{split}
&A \leftarrow \text{Καλλιτέχνης-Ομάδα} \bowtie
\sigma_{<\text{Τύπος} = \text{"Μουσική εκδήλωση"}>}\text{Εκδήλωση}
\\
&\Pi_{<\text{Όνομα\_εκδήλωσης, Ημερομηνία, Όνομα}>}A
\end{split}
\end{equation}

Ένα εξίσου σημαντικό ερώτημα  είναι η αναζήτηση μια εκδήλωσης βάση της
ημέρας διεξαγωγής. Είτε για μία συγκεκριμένη ημερομηνία και ώρα είτε
για ένα εύρος (\ref{eq4}).

\begin{equation}
  \label{eq4}
  \begin{split}
    &\Pi_{<\text{Όνομα\_εκδήλωσης, Ημερομηνία, Όνομα}>}(
    \sigma_{<\text{Ημερομηνία} = 23/11/2018>} \text{Εκδήλωση} \bowtie
    \text{Καλλιτέχνης-Ομάδα})
  \end{split}
\end{equation}

Πέρα από τα ερωτήματά όπου αφορά τα στοιχεία των εκδηλώσεων, ένα
σημαντικό ερώτημά, αναγκαίο για την ολοκλήρωση της παροχής υπηρεσιών
αγοράς εισιτήριων, είναι η προβολή όλων των χρηστών όπου έχουν
πραγματοποιήσει αγορά κάποιου εισιτηρίου για μια συγκεκριμένη εκδήλωση
(\ref{eq5}). Αυτό θα είναι διαθέσιμη μόνο στον χρήστη Διοργανωτής ο
οποίος συσχετίζεται με την εκάστοτε εκδήλωση.

\begin{equation}
  \label{eq5}
  \begin{split}
    &\Pi_{<\text{Ονοματεπώνυμο, Τύπος\_εισιτηρίου}>}(
    \sigma_{<\text{Κωδικός\_εκδήλωσης} = 42>} \text{Αγορά} \bowtie
    \text{Χρήστης})
  \end{split}
\end{equation}

Τέλος, παρουσιάζεται ένα ακόμα ερώτημα το οποίο αφορά τον χρήστη
Εγγεγραμμένος Χρήστης ο οποίος θα θελήσει να προβάλει όλες τις
αποθηκευμένες του εκδηλώσεις . (\ref{eq6}).


\begin{equation}
  \label{eq6}
  \begin{split}
    &A \leftarrow \sigma_{<\text{Κωδικός\_χρήστη} = 8055>}
    \text{Ενδιαφέρον} \bowtie \text{Εκδήλωση} \bowtie \text{Τοποθεσία}
    \bowtie \text{Καλλιτέχνης-Ομάδα} \\
    &\Pi_{<\text{Όνομα, Ημερομηνία, Ώρα\_έναρξης, Όνομα\_τοποθεσίας,
        Όνομα}>}A
  \end{split}
\end{equation}



% (Δώστε ενδεικτικά παραδείγματα χρήσιμων ερωτημάτων.)

% Παράδειγμα για τη FlightsDB:

% έστω οι σχέσεις:

% \begin{itemize}[noitemsep]
% \item CUSTOMER(cust\_id, firstname, lastname, phone, street, city,
%   zip)
% \item RESERVATION(flight\_id, date, cust\_id, ticket\_no, seat\_no)
% \end{itemize}

% Για μια πτήση (έστω την AA101) υποθέτουμε ότι ο/η αεροσυνοδός θα ήθελε
% να έχει τη λίστα των επιβατών μαζί με χρήσιμες πληροφορίες για το
% check in (id επιβάτη, αριθμός εισιτηρίου, θέση, όνομα και επώνυμο για
% κάθε επιβάτη). Εκτελούμε το παρακάτω ερώτημα:

% πticket\_no, seat\_no, cust\_id(σflight\_id=AA101(RESERVATION))
% πcust\_id, firstname, lastname(CUSTOMER)


%%% Local Variables:
%%% mode: latex
%%% TeX-master: "main"
%%% TeX-engine: xetex
%%% End:
