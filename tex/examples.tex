\section{Παραδείγματα}

\subsection{Παραδείγματα Πινάκων}

(Δώστε ενδεικτικά παραδείγματα εγγραφών για κάθε πίνακα της βάσης.)


manos 

Παράδειγμα για τον πίνακα Airport της FlightsDB:

\begin{tabular}{|p{3cm}|p{4cm}|p{3cm}|p{3cm}|}
  \hline
  airport\_code & name & city & country \\ \hline
  SKG & Makedonia & Thessaloniki & Greece \\ \hline
  ATH & Eleftherios Venizelos & Athens & Greece \\ \hline
  KVA & Megas Alexandros & Kavala & Greece \\ \hline
\end{tabular}
  
Εκτίμηση για τον αριθμό των εγγραφών: \textasciitilde 40000

\subsection{Παραδείγματα Ερωτημάτων}

(Δώστε ενδεικτικά παραδείγματα χρήσιμων ερωτημάτων.)

Παράδειγμα για τη FlightsDB:

έστω οι σχέσεις:

\begin{itemize}[noitemsep]
\item CUSTOMER(cust\_id, firstname, lastname, phone, street, city,
  zip)
\item RESERVATION(flight\_id, date, cust\_id, ticket\_no, seat\_no)
\end{itemize}

Για μια πτήση (έστω την AA101) υποθέτουμε ότι ο/η αεροσυνοδός θα ήθελε
να έχει τη λίστα των επιβατών μαζί με χρήσιμες πληροφορίες για το
check in (id επιβάτη, αριθμός εισιτηρίου, θέση, όνομα και επώνυμο για
κάθε επιβάτη). Εκτελούμε το παρακάτω ερώτημα:

πticket\_no, seat\_no, cust\_id(σflight\_id=AA101(RESERVATION))
πcust\_id, firstname, lastname(CUSTOMER)


%%% Local Variables:
%%% mode: latex
%%% TeX-master: "main"
%%% End:
